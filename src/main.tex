\documentclass[11p]{article}
% Packages
\usepackage{amsmath}
\usepackage{graphicx}
\usepackage{fancyheadings}
\usepackage[swedish]{babel}
\usepackage[
    backend=biber,
    style=authoryear-ibid,
    sorting=ynt
]{biblatex}
\usepackage[utf8]{inputenc}
\usepackage[T1]{fontenc}
%Källor
\addbibresource{references.bib}
\graphicspath{ {./images/} }

% Lite variabler
\def\email{samuel.viglundsson@ga.ntig.se}
\def\foottitle{PMmall}
\def\name{Samuel Viglundsson}

\title{PMmall \\ \small Gymnasiearbete}
\author{\namn}
\date{\today}

\begin{document}


% fixar sidfot
\lfoot{\footnotesize{\name \\ \email}}
\rfoot{\footnotesize{\today}}
\lhead{\sc\footnotesize\foottitle}
\rhead{\nouppercase{\sc\footnotesize\leftmark}}
\pagestyle{fancy}
\renewcommand{\headrulewidth}{0.2pt}
\renewcommand{\footrulewidth}{0.2pt}

% i Sverige har vi normalt inget indrag vid nytt stycke
\setlength{\parindent}{0pt}
% men däremot lite mellanrum
\setlength{\parskip}{10pt}

\maketitle

\section{Bakgrund}
Min frågestälnning är hur gamification(spelifiering) samt spel-baserad inlärning(game-based learning) fungerar i praktiken.


Gamification är att ta delarna som gör spel roliga in i lärande t.ex ha ett narrativ/story för det man gör, poäng system för att göra uppgifter, levlar och bossbattles, direkt feedback av arbete, achievements, tävla mot andra spelare eller dig själv. Jag kommer dels se vad rådande kunskapsläget är gällande gamficaitions effektvietet, dels se hur lärare och sånt känner om gamficifaiton och om det är någonting de använder i lärandet, och sist tänkte jag hålla något experiment där klassen(antaglig vår) får lära sig något nytt och några får lära sig med hjälp av gamfiication och andra inte för att se om det är någon skillnad. Inte ett perfekt experiment då inlärningstiden är så kort men det funkar.

\section{Metod}
Min metod är

\section{Referenser}
Då gamification som koncept är ganska nytt finns inte speciellt många studier som pratar om just gamficiations effektivetet som jag har förståt det. Jag ska läsa delar av 'Det Spelifierade klassrummet' av Adam Palmquist för refrenser om vad exakt gamifictaion är och hur man använder det. Jag ska även intervjua en lärare på dragonen som specialiserar sig på gamification: Niclas Lind heter han, Niclas har jättemycket erfarenhet med gamification. Jag ska spela in intervjuvn(givetviss kollade jag så det är okej innan, det är okej) och kan skicka kopia till er sen om ni vill. Jag tror att boken och Niclas kunskap kommer att räcka som källor för de mesta, men kan hända att jag tar upp någonting mer om jag känner det skulle behövas.
\section{Annat som kan vara bra att veta}
Det är lätt att skriva matematik i \LaTeX

\begin{equation}
    F = G \frac{M m}{r^2}
    \label{eq:grav}
\end{equation}

Ekvation (\ref{eq:grav}) känner ni igen...
\clearpage
\subsection{En underrubrik}
    \begin{figure}[!h]
        \includegraphics[width=0.8\textwidth]{accelerationTime.png}
        \caption{Acceleration-tid diagram. Källa: \textcite{Fraenkel}}
        \label{fig:acc}
    \end{figure}

Acceleration-tiddiagram (se figur \ref{fig:acc})

\printbibliography

\end{document}
